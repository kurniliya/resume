%-----------------------------------------
% Ilya Kurnosov
% Resumé
%
% URL: http://github.com/kurniliya/resume
% This document is covered under the Creative Commons Attribution 3.0 Unported License
%-----------------------------------------

%!TEX TS-program = xelatex
%!TEX encoding = UTF-8 Unicode

\documentclass[14pt,a4paper,final]{moderncv}
\moderncvtheme[burgundy]{classic}
\moderncvicons{awesome}

% DOCUMENT LAYOUT
\usepackage[scale=0.85]{geometry}
\settowidth{\hintscolumnwidth}{Abc 1234 -- Cde 5678}
\AtBeginDocument{\recomputelengths}

% FONTS
\usepackage[utf8]{inputenc}
\usepackage[sfdefault]{FiraSans}

% SYMBOLS
\newcommand*\skypephonesymbol{{\small\faSkype}~}

% Personal Information 
\firstname{Ilya}
\familyname{Kurnosov}
\phone[mobile]{+7 (926) 550-3712}
\phone[skype]{i.kurnosov}
\email{kurniliya@gmail.com}
\social[linkedin]{ilyakurnosov}
\social[twitter]{kurniliya}
\social[github]{kurniliya}

\begin{document}

\maketitle

\section{Experience} 
  \cventry{Aug~2015 -- Present}{Senior Test and QA Analyst}{Deutsche Bank}{Moscow, Russia}{}{
    Autotesting of server side components of the bank's Rapid FX trading platform (Linux OS, C++, Java, kdb+):
    \begin{itemize}
      \item Development of in-house test automation framework (Java, Spring, Python).
      \item Maintenance and expansion of automated test library.
    \end{itemize}
  }
  \cventry{Mar~2014 -- Aug~2015}{Senior SQA Engineer}{Align Technology}{Moscow, Russia}{}{
    Testing of software for iTero intraoral scanners (Windows 8, Visual C++, C\#, WPF, SQLite):
    \begin{itemize}
      \item Worked as part of Scrum-But project team developing next generation of company's intraoral scanners (iTero Element).
      \item Together with the team's another Sr. SQA engineer owned all the SQA aspects in the team: effort estimation,
          test design and execution, preparation of reports for management, etc.
      \item Participated in the team's code reviews (Visual C++, Python).
      \item Defined and implemented from scratch an approach and framework for functional test automation
          (Python, PyUnit, nose) leveraging API exposed by the product (ZeroMQ, Protocol Buffers).
    \end{itemize}
  }
  \cventry{Apr~2010 -- Feb~2014}{Senior Test Automation Engineer}{Epicor Software Corp}{Moscow, Russia}{}{
    Maintenance of iScala Automation Tool (Visual Basic 6, Visual C++, COM, ADO) and other supporting homegrown tools (VB6, C\#, PowerShell):
    \begin{itemize}
      \item Development of patches for troubles reported by autotest engineers and external customers.
      \item Development of new features requested by autotest engineers.
      \item Coordination with iScala platform team.
      \item Development of various internal reports to support autotest team's operations.
    \end{itemize}
    Autotesting of Epicor iScala ERP product (Windows OS/MS SQL Server DB/VB6, VC++, COM+) using homegrown automation tools:
    \begin{itemize}
      \item Maintenance of operational regression script library.
      \item Development of new automated test scripts for product's new features.
      \item Analysis of existing automated test scripts.
      \item Preparation of test environments.
      \item Analysis of test execution results.
      \item Hardware planning for test automation effort.
    \end{itemize}
    Autotesting of Epicor iScala integration with MS Dynamics CRM 4 using Rational Robot.
  }
  \cventry{Nov~2007 -- Mar~2010}{Software Test Engineer}{CBOSS}{Moscow, Russia}{}{
    Testing of CBOSS 4 billing \& CRM product (Oracle 10g DB/PL SQL/Oracle Forms/Java):
    \begin{itemize}
      \item Testing of product's new features (development of test scenarios and test cases, execution of the tests).
      \item Execution of regression tests for the product based on existing scenarios.
      \item Maintenance of regression scenarios.
      \item Load and performance testing.
    \end{itemize}
  }
  \cventry{Apr~2007 -- Oct~2007}{Engineer at Department of Applied Mathematics}{Moscow Engineering Physics Institute}{Moscow, Russia}{}{
    Research in numerical and analytical methods of solution of partial differential equations.
  }

\section{Languages}
\cvlanguage{English}{Fluent}{}
\cvlanguage{Russian}{Native}{}

\section{Education}
\cventry{2003 -- 2011}{Applied mathematics and computer science}{Moscow Engineering Physics Institute}{Moscow, Russia}{Master's degree}{}

\section{Extra}
  \cvlistitem{Contributor to open source projects -- see \url{https://github.com/kurniliya}.}

\end{document}
